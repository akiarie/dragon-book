\documentclass[11pt,oneside,a4paper]{article}

%----------------------------------------------------------------------------------------
%	PACKAGE INCLUSIONS
%----------------------------------------------------------------------------------------

%\usepackage[lf]{Baskervaldx} % lining figures
%\usepackage[bigdelims,vvarbb]{newtxmath} % math italic letters from Nimbus Roman

\usepackage{graphicx}
\usepackage{color}

\let\Bbbk\relax
\usepackage{amsfonts}
\let\openbox\relax
\usepackage{amsmath}
\usepackage{amssymb}
\usepackage{amsthm}
\usepackage{mathrsfs}
\usepackage{minted}

\begin{document}

Let $P$ stand for the conjunction of the conditions
\begin{itemize}
	\item $P1$: For all integers $i$ such that $1 \le i \le s$ we have  $f(i) = F(i)$
	\item $P2$: $t = f(s)$,
\end{itemize}
and assume that $P$ holds prior to execution of the following line:
\[
	\text{while}~(t > 0~\text{and}~b_{s+1} \ne b_{t+1})~t = f(t);
\]
Our aim is to show that the line will terminate and upon termination we will have that
\[ \big( b_{s+1} \ne b_{t+1}~\text{and}~F(s+1) = t = 0 \big) \]
or
\[ \big( b_{s+1} = b_{t+1}~\text{and}~F(s+1) = t+1 \big). \]
Let $R = (R1~\text{or}~R2)$ stand for this disjunction, respectively.

Termination is easy to see: since $f(t) < t$, we will eventually get $t = 0$ which will ensure
termination.

Let $Q = (Q1~\text{and}~Q2)$ for
\begin{itemize}
	\item $Q1$: $F(s+1) \le t + 1$
	\item $Q2$: $b_1b_2\ldots{}b_t$ is a proper prefix and a suffix of $b_1b_2\ldots{}b_s$.
\end{itemize}
Note that upon initialisation we have $t = f(s) = F(s)$ from $P$, and that in all cases
\[ F(i+1) \le F(i) + 1, \]
hence $Q1$ must hold. Since $t = F(s)$ we also get that $b_1\ldots{}b_t$ is a proper prefix and a
suffix of $b_1\ldots{}b_s$, so $Q$ is established.

Now assume $Q$. 
When $t = f(t)$ is executed, we must initially have $t > 0$ and $b_{s+1}\ne{}b_{t+1}$. 
Since from $Q2$ we know $b_1\ldots{}b_t$ is a proper prefix and a suffix of $b_1\ldots{}b_s$, the
fact that $b_{s+1}\ne{}b_{t+1}$ tells us that $F(s + 1)$ is strictly less than $t+1$, indeed that
\[ F(s + 1) \le f(t) + 1 \]
because $t > 0$. $Q$ will therefore remain invariant under the assignment, because
$b_1\ldots{}b_{f(t)}$ will be a proper prefix and suffix of $b_1\ldots{}b_s$ since it is such with
respect to $b_1\ldots{}b_t$.

Since $Q$ is invariant, when the loop terminates we will have $Q$ and
\[ (t = 0)~\text{or}~(b_{s+1} = b_{t+1}). \]
If $t = 0$, then $Q1$ implies $F(s+1) \le 1$.
If $b_{s+1} \ne b_{t+1}$ then evidently $F(s+1) = 0$, so $R1$ holds.
If $b_{s+1} = b_{t+1}$ then clearly $F(s+1) = 1 = t + 1$ since $t = 0$. Thus $R2$ holds.
Hence we have $R$ established.

On the other hand, if $b_{s+1} = b_{t+1}$, since from $Q2$ we know $b_1\ldots{}b_t$ is a proper
prefix and a suffix of $b_1\ldots{}b_s$, it follows that $b_1\ldots{}b_{t}b_{t+1}$ sustains the same
relation to $b_1\ldots{}b_{s}b_{s+1}$. But then, by definition, we must have \[t + 1 \le F(s+1),\]
since $F(s+1)$ is the \textit{longest} proper prefix and suffix of $b_1\ldots{}b_{s}b_{s+1}$.
This last inequality combines with $Q1$ to establish $R2$, which implies $R$.

\end{document}
